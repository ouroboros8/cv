%%%%%%%%%%%%%%%%%%%%%%%%%%%%%%%%%%%%%%%%%
% Plasmati Graduate CV
% LaTeX Template
% Version 1.0 (24/3/13)
%
% This template has been downloaded from:
% http://www.LaTeXTemplates.com
%
% Original author:
% Alessandro Plasmati (alessandro.plasmati@gmail.com)
%
% License:
% CC BY-NC-SA 3.0 (http://creativecommons.org/licenses/by-nc-sa/3.0/)
%
% Important note:
% This template needs to be compiled with XeLaTeX.
% The main document font is called Fontin and can be downloaded for free
% from here: http://www.exljbris.com/fontin.html
%
%%%%%%%%%%%%%%%%%%%%%%%%%%%%%%%%%%%%%%%%%

%========================================================================================
%	PACKAGES AND OTHER DOCUMENT CONFIGURATIONS
%========================================================================================

\documentclass[a4paper,10pt]{article} % Default font size and paper size

\usepackage{fontspec} % For loading fonts
\defaultfontfeatures{Mapping=tex-text}
\setmainfont[
  SmallCapsFont={Fontin Sans}
]{Ubuntu} % Main document font

\usepackage{fontspec,url,parskip} % Formatting packages

\usepackage[usenames,dvipsnames]{xcolor} % Required for specifying custom colors

% To reduce the height of the top margin uncomment:
\usepackage[top=1.5cm,bottom=1.5cm,left=2.5cm,right=2.5cm]{geometry}

\usepackage{hyperref} % Required for adding links	and customizing them
\definecolor{linkcolour}{rgb}{0,0.2,0.6} % Link color
\hypersetup{colorlinks,breaklinks,urlcolor=linkcolour,linkcolor=linkcolour} % Set link colors throughout the document

\usepackage{titlesec} % Used to customize the \section command
\newcommand*{\justifyheading}{\raggedright}
\titleformat{\section}{\Large\scshape\justifyheading}{}{0em}{}[\titlerule] % Text formatting of sections
\titlespacing{\section}{0pt}{3pt}{3pt} % Spacing around sections

\usepackage{tabularx}
\usepackage{array}
\usepackage{enumitem}

\usepackage[british]{babel}

\begin{document}

\pagestyle{empty} % Removes page numbering
\setlist{after=\vspace{-1\baselineskip}}
% No hyphens plz
\hyphenpenalty=10000
\exhyphenpenalty=10000

%========================================================================================
%	NAME AND CONTACT INFORMATION
%========================================================================================

\centering{\huge \textsc{Daniel Strong}\bigskip}\\ % Your name
\hrule
\footnotesize
{\begin{tabularx}{\textwidth}{lXr}
\href{mailto:dstrong@glyx.co.uk}{dstrong@glyx.co.uk}                    & & {41 Upper Chorlton Road}\\
\href{https://github.com/ouroboros8}{github.com/ouroboros8}             & & {Manchester}\\
\href{https://www.linkedin.com/in/dstrong52}{linkedin.com/in/dstrong52} & & {M16 7RW} \\
{0758 315 81 81}                                                        & &
\end{tabularx}}

%========================================================================================
%	Summary
%========================================================================================
\begin{flushleft}
I am a perpetually curious generalist, seeking roles in diverse environments
which will broaden my understanding of all aspects of software development,
design and delivery. I am passionate about software quality and prefer
test-driven and data-driven development methodologies, continuous iteration and
constant feedback.

I have a particular interest in continuous delivery and release engineering, a
critical but often invisible part of agile software lifecycles which impacts
security, observability and velocity.

I strive to be inclusive, continuously improve, and keep it simple.
\end{flushleft}

%========================================================================================
%	WORK EXPERIENCE
%========================================================================================

\section{\textsc{Work Experience}}
\centering
\begin{tabularx}{\textwidth}{r|X}

\textbf{Dec 2019 -- Present}        & \textbf{Senior Systems Engineer} at the BBC \\
                                    & \emph{Manchester, UK} \\
                                    & \\
                                    & \footnotesize
    {At the BBC I worked in BBC R\&D's Future Experience Technologies (FXT)
        section, helping research teams streamline their development and
        deployment practices.
    \begin{itemize}
        \item Built and supported backend infrastructure and monitoring
            (CloudWatch and InfluxDB) for the
            \href{https://www.bbc.co.uk/together}{BBC Together} public trial.
        \item TODO...
    \end{itemize}}\\

\multicolumn{2}{r}{} \\ %----------------------------------------------------------------

\textbf{Nov 2017 -- Dec 2019}       & \textbf{Platform Engineer} at The Co-operative Group \\
                                    & \emph{Manchester, UK} \\
                                    & \\
                                    & \footnotesize
    {I worked across Co-op's Digital division, first as part of a dedicated
        platform team, then as an embedded expert in the Membership team and
        later on Co-op Health.
    \begin{itemize}
    \item Reduced automated deployment times for Co-op Membership services by a
        factor of 10 using preconfigured AMIs provisioned with Packer.
    \item Ran spikes on API rate limiting, weighing up the advantages of rate
        limiting calls to AWS API Gateway with a python Lambda implementing the
        token-bucket algorithm against running nginx in sidecar containers
        behind the gateway, as well as evaluating a third party cloud WAF
        appliance providing rate limiting on the network edge.
    \item Drove efforts to refactor and simplify the templated terraform set up
        used by delivery teams across digital using a flatter and more cohesive
        module structure which minimises config drift between environments.
    \item Rebuilt the Co-op Health APIs Jenkins CI/CD pipeline with close
        regard to ease of use for developers, consolidating multiple step
        release process into one single-click declarative pipeline per
        environment.
    \end{itemize}}\\

\multicolumn{2}{r}{} \\ %----------------------------------------------------------------

\textbf{Nov 2016 -- Oct 2017}        & \textbf{WebOps Engineer} at HM Revenue \& Customs \\
                                    & \emph{London, UK} \\
                                    & \\
                                    & \footnotesize
    {As part of HMRC Digital's WebOps Engineering team, I built, maintained and
        extended the Multi-channel Digital Tax Platform (MDTP), the
        docker-based PaaS which hosts HMRC's digital services.
    \begin{itemize}
    \item Led the reimplementation of MDTP on AWS and reduced the size of the
        infrastructure codebase (in lines of code) by a factor of 10, massively
        reducing architectural complexity and simplifying platform maintenance.
    \item Trained and upskilled new joiners, allowing the migration to continue
        despite a the loss of 75\% of our engineering team due to tax code
        changes.
    \item Collaborated with development teams to ensure a smooth, zero-downtime
        cutover to AWS, minimising disruption both to services' release cycles
        and to our customers.
    \end{itemize}}\\

\multicolumn{2}{r}{} \\ %----------------------------------------------------------------

\textbf{Nov 2015 -- Nov 2016}       & \textbf{DevOps Support Engineer} at HM Revenue \& Customs \\
                                    & \emph{London, UK} \\
                                    & \footnotesize
    {\begin{itemize}[after=\vspace{-2\baselineskip}]
    \item Streamlined incident response and resolution by establishing a triage
    rota and work-in-progress limits, resulting in the elimination of a backlog
    which had fluctuated between 20 and 60 tickets since I joined.
    \item Extended monitoring to release-critical pre-production environments,
        leading to a marked reduction in disruption to service development
        teams.
    \item Implemented predictive monitoring for simple trends like disk usage
    using linear regression.
    \end{itemize}}\\

\end{tabularx}

%========================================================================================
%	COMPUTER SKILLS
%========================================================================================

\section{\textsc{Skills}}
\begin{tabularx}{\textwidth}{p{0.3cm}|p{4cm}X}
%----------------------------------------------------------------------------------------
\multicolumn{2}{l}{\textbf{DevOps}} & \\
\multicolumn{3}{c}{} \\ % - - - - - - - - - - - - - - - - - - - - - - - - - - - - - - - -
   & Linux administration:               & Containerisation, TCP/IP, configuration management, RBAC and SELinux \\
     \\
   & Infrastructure as Code:             & Terraform, Packer \\
     \\
   & Logging and monitoring:             & Elasticsearch, Logstash, Kibana (ELK stack); Sensu, Graphite, collectd \\
     \\
   & Continuous Integration:             & Jenkins X, GitHub Actions, GitLabCI, Travis, CircleCI \\
     \\
   & Serverless:                         & Object stores, Functions-as-a-Service (Lambda), AWS RDS, AWS API Gateway \\
     \\
   & Networking:                         & Public cloud networking, load balancing, NAT and reverse proxying \\
     \\
   & Identity and permissions:           & Public cloud IAM, OAuth 2.0, secret management \\
\multicolumn{3}{c}{} \\ %----------------------------------------------------------------
\multicolumn{2}{l}{\textbf{Programming Languages}}      & \\
\multicolumn{3}{c}{} \\ % - - - - - - - - - - - - - - - - - - - - - - - - - - - - - - - -
  &  Advanced:                           & Python, Bash \\
     \\
  &  Basic                               & JavaScript, Go, Rust, Clojure \\
\end{tabularx}

%========================================================================================
%	LANGUAGES
%========================================================================================
\section{Interests}
\raggedright
In my spare time I play video games and board games, read widely, and help out
with tech stuff at \href{https://nisei.net}{NISEI}, a fan-run nonprofit
volunteer collective supporting the out of print Android: Netrunner card game.

%
%========================================================================================
%	INTERESTS AND ACTIVITIES
%========================================================================================
%
\section{\textsc{Languages}}
\begin{tabular}{rl}
 \textbf{English:} & Native \\
 \textbf{Spanish:} & Fluent \\
 \textbf{Catalan:} & Fluent \\
 \textbf{Italian:} & Conversational \\
 \textbf{French:}  & Conversational
\end{tabular}

\end{document}
